% E&M Physics LaTeX Preamble - Core Laws & Equations

% Basic packages
\usepackage{amsmath}    % Enhanced math environments
\usepackage{amssymb}    % Additional math symbols
\usepackage{mathtools}  % Extensions to amsmath
\usepackage{physics}    % Physics notation (vectors, derivatives, etc.)
\usepackage{siunitx}    % SI units formatting
\usepackage{graphicx}   % For including figures
\usepackage{float}      % Better figure placement control
\usepackage{hyperref}   % Clickable references and links
\usepackage{bm}         % Bold math symbols for vectors

% Vector notation
\renewcommand{\vec}[1]{\boldsymbol{\mathbf{#1}}}  % Bold vectors
\newcommand{\uvec}[1]{\hat{\mathbf{#1}}}         % Unit vectors

% Common vectors and symbols
\newcommand{\E}{\vec{E}}       % Electric field
\newcommand{\B}{\vec{B}}       % Magnetic field
\newcommand{\D}{\vec{D}}       % Electric displacement field
\newcommand{\J}{\vec{J}}       % Current density
\newcommand{\A}{\vec{A}}       % Vector potential
\newcommand{\p}{\vec{p}}       % Dipole moment
\newcommand{\Pv}{\vec{P}}      % Polarization
\newcommand{\M}{\vec{M}}       % Magnetization
\newcommand{\F}{\vec{F}}       % Force
\newcommand{\eps}{\varepsilon_0}  % Vacuum permittivity
\newcommand{\perm}{\mu_0}      % Vacuum permeability
\newcommand{\pot}{\phi}        % Scalar potential




% Integrals
\newcommand{\volint}{\iiint}                      % Volume integral
\newcommand{\surfint}{\oiint}                     % Closed surface integral
\newcommand{\lineint}{\oint}                      % Closed line integral

% ======= ELECTROSTATICS LAWS ======= %

% Coulomb's Law
\newcommand{\coulomblaw}{
\F = \frac{1}{4\pi\eps}\frac{q_1 q_2}{r^2}\uvec{r}
}

% Electric field of a point charge
\newcommand{\pointchargefield}{
\E = \frac{1}{4\pi\eps}\frac{q}{r^2}\uvec{r}
}

% Gauss's Law
\newcommand{\gausslaw}{
\divg \E = \frac{\rho}{\eps}
}

% Integral form of Gauss's Law
\newcommand{\gausslawint}{
\surfint \E \cdot d\vec{a} = \frac{Q_{enc}}{\eps}
}

% Electric potential
\newcommand{\electricpotential}{
V = \frac{1}{4\pi\eps}\int \frac{\rho(\vec{r'})}{|\vec{r}-\vec{r'}|} d\tau'
}

% Relationship between potential and electric field
\newcommand{\potentialfield}{
\E = -\grad V
}

% Poisson's Equation
\newcommand{\poisson}{
\laplacian V = -\frac{\rho}{\eps}
}

% Laplace's Equation
\newcommand{\laplace}{
\laplacian V = 0
}

% Electric dipole moment
\newcommand{\dipolemoment}{
\p = q\vec{d}
}

% Electric field of a dipole (far field)
\newcommand{\dipolefield}{
\E = \frac{1}{4\pi\eps}\frac{3(\p \cdot \uvec{r})\uvec{r} - \p}{r^3}
}

% Energy density in electric field
\newcommand{\electricenergy}{
u_E = \frac{1}{2}\eps |\E|^2
}

% Dielectric materials relation
\newcommand{\dielectric}{
\D = \eps_0 \E + \Pv = \eps \E
}

% ======= MAGNETOSTATICS LAWS ======= %

% Biot-Savart Law
\newcommand{\biotsavart}{
\B = \frac{\perm}{4\pi}\int \frac{I d\vec{l} \times \uvec{r}}{r^2}
}

% Magnetic field of a point charge
\newcommand{\movingchargefield}{
\B = \frac{\perm}{4\pi}\frac{q\vec{v}\times\uvec{r}}{r^2}
}

% Ampere's Law
\newcommand{\amperelaw}{
\curl \B = \perm \J
}

% Integral form of Ampere's Law
\newcommand{\amperelawint}{
\lineint \B \cdot d\vec{l} = \perm I_{enc}
}

% Magnetic vector potential
\newcommand{\vectorpotential}{
\A = \frac{\perm}{4\pi}\int \frac{\J(\vec{r'})}{|\vec{r}-\vec{r'}|} d\tau'
}

% Relationship between vector potential and magnetic field
\newcommand{\potentialB}{
\B = \curl \A
}

% Magnetic dipole moment
\newcommand{\magneticdipolemoment}{
\vec{m} = I\vec{a}
}

% Magnetic field of a dipole (far field)
\newcommand{\magneticdipolefield}{
\B = \frac{\perm}{4\pi}\frac{3(\vec{m} \cdot \uvec{r})\uvec{r} - \vec{m}}{r^3}
}

% Energy density in magnetic field
\newcommand{\magneticenergy}{
u_B = \frac{1}{2\perm}|\B|^2
}

% Magnetic materials relation
\newcommand{\magnetic}{
\B = \perm_0(\H + \M) = \perm \H
}

% Lorentz force law
\newcommand{\lorentzforce}{
\F = q(\E + \vec{v} \times \B)
}

% ======= MAXWELL'S EQUATIONS ======= %

% Maxwell's equations (differential form)
\newcommand{\maxwelldiff}{
\begin{align}
\divg \D &= \rho \\
\divg \B &= 0 \\
\curl \E &= -\frac{\partial \B}{\partial t} \\
\curl \H &= \J + \frac{\partial \D}{\partial t}
\end{align}
}

% Maxwell's equations (integral form)
\newcommand{\maxwellint}{
\begin{align}
\surfint \D \cdot d\vec{a} &= Q_{enc} \\
\surfint \B \cdot d\vec{a} &= 0 \\
\lineint \E \cdot d\vec{l} &= -\frac{d}{dt}\int \B \cdot d\vec{a} \\
\lineint \H \cdot d\vec{l} &= I_{enc} + \frac{d}{dt}\int \D \cdot d\vec{a}
\end{align}
}

% Electromagnetic wave equation
\newcommand{\waveequation}{
\nabla^2 \E = \frac{1}{c^2}\frac{\partial^2 \E}{\partial t^2}
}

% Wave velocity in vacuum
\newcommand{\wavespeed}{
c = \frac{1}{\sqrt{\eps_0 \perm_0}}
}

% Poynting vector
\newcommand{\poynting}{
\vec{S} = \frac{1}{\perm_0}(\E \times \B)
}